%%%%%%%% ICML 2018 EXAMPLE LATEX SUBMISSION FILE %%%%%%%%%%%%%%%%%

\documentclass{article}

% Recommended, but optional, packages for figures and better typesetting:
\usepackage{microtype}
\usepackage{graphicx}
\usepackage{hyperref}
\usepackage{cleveref}
\usepackage{bm}
\usepackage{amsthm}
\usepackage{amsfonts}
\usepackage{subfigure}
\usepackage{stmaryrd}
\usepackage{booktabs} % for professional tables

% hyperref makes hyperlinks in the resulting PDF.
% If your build breaks (sometimes temporarily if a hyperlink spans a page)
% please comment out the following usepackage line and replace
% \usepackage{icml2018} with \usepackage[nohyperref]{icml2018} above.
\usepackage{amsmath}

% Attempt to make hyperref and algorithmic work together better:
\newcommand{\theHalgorithm}{\arabic{algorithm}}

\newcommand{\norm}[1]{\left\lVert#1\right\rVert}
\newcommand{\abs}[1]{\left|#1\right|}
\newcommand{\diag}[1]{\text{diag}\left(#1\right)}
\newcommand{\intint}[1]{\left\llbracket#1\right\rrbracket}
\newtheorem{proposition}{Proposition}[section]

% Use the following line for the initial blind version submitted for review:
\usepackage{icml2018}
\newcommand{\srm}[1]{\textcolor{red}{{\bf Sam:} #1}}
\newcommand{\ra}[1]{\textcolor{blue}{{\bf ra:} #1}}
\newcommand{\gl}[1]{\textcolor{violet}{{\bf Gl:} #1}}
\newcommand{\mpv}[1]{\textcolor{green}{{\bf MV:} #1}}
% If accepted, instead use the following line for the camera-ready submission:
%\usepackage[accepted]{icml2018}

% The \icmltitle you define below is probably too long as a header.
% Therefore, a short form for the running title is supplied here:
\icmltitlerunning{Learning Neural Network size with ShrinkNets}

\begin{document}

\twocolumn[
\icmltitle{Learning Network Size while Training with ShrinkNets}

% It is OKAY to include author information, even for blind
% submissions: the style file will automatically remove it for you
% unless you've provided the [accepted] option to the icml2018
% package.

% List of affiliations: The first argument should be a (short)
% identifier you will use later to specify author affiliations
% Academic affiliations should list Department, University, City, Region, Country
% Industry affiliations should list Company, City, Region, Country

% You can specify symbols, otherwise they are numbered in order.
% Ideally, you should not use this facility. Affiliations will be numbered
% in order of appearance and this is the preferred way.
\icmlsetsymbol{equal}{*}

\begin{icmlauthorlist}
\icmlauthor{Guillaume Leclerc}{mit,epfl}
\icmlauthor{Raul Castro Fernandez}{mit}
\icmlauthor{Manasi Vartak}{mit}
\icmlauthor{Martin Jaggi}{epfl}
\icmlauthor{Samuel Madden}{mit}
\end{icmlauthorlist}

\icmlaffiliation{mit}{Computer Science and Artificial Intelligence Laboratories, Massachusetts Institute of Technology, Cambridge, Massachusetts, USA}

\icmlaffiliation{epfl}{Swiss Federal Institute of Technology, Lausanne, Switzerland}

\icmlcorrespondingauthor{Guillaume Leclerc}{leclerc@mit.edu}

% You may provide any keywords that you
% find helpful for describing your paper; these are used to populate
% the "keywords" metadata in the PDF but will not be shown in the document
\icmlkeywords{Machine Learning, ICML}

\vskip 0.3in
]

% this must go after the closing bracket ] following \twocolumn[ ...

% This command actually creates the footnote in the first column
% listing the affiliations and the copyright notice.
% The command takes one argument, which is text to display at the start of the footnote.
% The \icmlEqualContribution command is standard text for equal contribution.
% Remove it (just {}) if you do not need this facility.

%\printAffiliationsAndNotice{}  % leave blank if no need to mention equal contribution
\printAffiliationsAndNotice{} % otherwise use the standard text.

\begin{abstract}
  Let's write the abstract at the end
\end{abstract}

\section{Tentative outline}

% \begin{itemize}
  % \item We want to figure out what is the proper network size
  % \item If we had a oversized network and for each neuron we would have an on/off switch, we could optimize the state of each switch to achieve any size/accuracy tradeof (we can prove that, do we care ?)
  % \item Solving such problem is NP-Hard
  % \item We could approximate the binary switch by relaxing them and doing L1 loss instead of L0 loss
  % \item This is the definition of Shrinknets
  % \item group sparsity tries to achieve a similar goal but with a different formulation
  % \item How are they related ?
  % \item If we add a specific constraint on our formulation we obtain the group sparsity one (proven)
  % \item We conclude that without this constraint our formulation has more degree of freedom
  % \item What happen when we drop this constraint
  % \item The problem becomes non-convex and without a global minimum
  % \item Not having a global minimum is bad, how can we solve that ?
  % \item Adding an extra regularization parameter allow us to have a global minimum (a lot of them actually)
  % \item Now we have a framework to switch on and of neurons
  % \item We are still doing useless computation for neurons that will stay off forever, if we pruned them then we would speed up training
  % \item Define the strategy to detect "forever dead" neurons
  % \item (should we talk about the software architecture ?)
  % \item now we evaluate the system
  % \item First we show that our system has indeed more freedom than Group sparsity by showing that for any given amount of sparsity it can fit closer than the previous method
  % \item On the same problem we show that removing neurons on the fly does not dramatically reduce accuracy (we might not be able to see any difference in performance for a linear/logistic regression though)
  % \item We show that works with Bigger networks (multi layers perceptrons and CNNs), describe the training dynamics, discuss the shape we obtain
  % \item Show that it is easier to do hyper-parameter optimization on the regularization parameter that the size of each layer. I think a nice experiment would be to plot the evolution of the final size of a two layer network (so going from a lambda to two sizes). And plot the trajectory in the "size space" as we decrease lambda. It would show how the system trades off the budget between the two layer in a nice and visual way
  % \item Then we evaluate performance
% \end{itemize}

% \section{Outline2}
\begin{itemize}
  \item Introduction
  \begin{itemize}
    \item Finding an appropriately sized network is challenging
    \item ML practitioners spend a large amount of time tuning the size of layers
    in a network in order to get the best possible accuracy.
    \item Many techniques have been proposed for hyperparam opt but these
    are computationally expensive and take long to train
    \item We propose ShrinkNets, an approach to tune the size of the network 
    as it is being trained without incurring the significant costs of 
    hyperparameter optimization. 
    \item Thus, our contribution are: ...
  \end{itemize}
  \item Our Approach
  \begin{itemize}
    \item Assign an on/off switch to each neuron. But this is np-hard, so we
    consider a relaxation
    \item Formulation
    \item Relationship to sparsity
    \item Strategies to kill neurons (relation to theory above?)
    \item Implementation details
  \end{itemize}
  \item Related Work
  \begin{itemize}
    \item Hyperparameter optimization: random, bayesian opt, bandit methods
    \item distillation techniques
    \item post-training compression techniques
    \item group sparsity, non-parametric neural networks
    \item training dynamics paper: first overfitting and then randomization?
  \end{itemize}
  \item Experiments
  \begin{itemize}
    \item Accuracy obtained by Shrinknets
    \item Time taken to reach that accuracy compared with other hyperopt methods
    \item Characterize method wrt params
    \item Other experiments  
  \end{itemize}
  \item Discussion
  \begin{itemize}
    \item Where does the method shine?
    \item Side-effects of method: smaller networks, time to train
    \item Potential extensions/limitations
  \end{itemize}
  \item Conclusion
\end{itemize}

\section{Introduction}

One of the key determinants of network accuracy is the shape of the network, 
i.e., the number of layers, number of neurons per layer, and connections between
layers.
An under-sized network is likely to have low accuracy because of insufficient 
capacity while an over-sized network is difficult to train due to additional 
parameters and is computationally inefficient during trainging as well as
inference.
% A sub-optimal network shape can lead to low accuracy .
%Although the hyperparameters related wo network size can dramatically affect
% performance, there is no reliable way to efficiently set them.
Consequently, many techniques have been proposed to determine the optimal size of
a neural network.
This problem is also referred to as {\it hyperparameter optimization}.
The most popular techniques include hyperparameter optimization strategies 
ranging from random search~\cite{paper-on-random-is-good-enough}, 
what-is-this-paper~\cite{Bengio2012a}, 
meta-gradient descent~\cite{Pedregosa2016},
Gaussian processes~\cite{Bergstra2011a} etc.
All of these techniques without exception require a compute-intensive search of 
parameter space and often the training of many tens or hundreds of models.
As a result, tuning models via hyperparameter optimization takes many times 
longer than the training of a single model and also requires more compute power
for the training of every model with different model shape.

In this paper we present a novel method to automatically find an appropriate
network size, drastically reducing optimization time. 
The key idea is to learn the right network size at the same time that
the network is learning the main task.
Our strategy, called \textbf{ShrinkNets}, is blah blah \mpv{Short blurb about the technique}
% For example, for an image classification task, with our approach we can 
% provide the training data to a network—without sizing it a priori—and expect 
% to end up with a network that has learned to classify images with an accuracy
% similar to a the best manually engineered network. 
Our approach has two main benefits. 
First and foremost, we no longer need to choose a network size before training.
We can merely set an initial size for the network and then the algorithm will
determine the best network that is smaller or equal in size to the inital size.
Second, our technique trains a single model as opposed to the hundreds trained
during hyperparameter optimization, as a result, we can find the best model 
faster as well as with less computational overhead.

Thus, our contributions are as follows: 
(a) we propose a novel technique based on dynamically switching neurons on/off 
to learn network size as the network is being trained. 
(b) we show that our technique is a relaxation\mpv{?} of group sparsity and 
prove \mpv{fill in}. 
(c) we demonstrate the efficacy of our technique on CNNs and FCNs where 
ShrinkNets finds networks within +/-X\% of best hand-crafted accuracy in XX\% of
training time compared to competing hyperparameter optimization methods.
(d) we also demonstrate that ShrinkNets can achieve this accuracy with only YY\% 
of neurons.

\section{Our Approach}

\begin{itemize}
  \item Assign an on/off switch to each neuron. But this is np-hard, so we
  consider a relaxation
  \item Formulation
  \item Relationship to group sparsity
  \item Strategies to kill neurons (relation to theory above?)
  \item Implementation details
\end{itemize}

Our approach has two main challenges. First, we need a way to
dynamically size the network during training. Second, we need a
loss function that optimizes for the additional task of sizing, without
deteriorating the learning performance of the main task. Our
approach, called ShrinkNets, copes with both challenges.

During training, our approach starts
with an explicitly over-sized network. As training progresses, we learn
which neurons are not contributing to learning and remove them
dynamically, effectively shrinking the network. This method requires two key components: first, we need a way to identify neurons that are not contributing to the
learning process, and second we need a way to balance the network size and the
generalization capability for the main task. We introduce a new \textsf{Filter} 
\mpv{Choose a different name if possible bc this can be confused w/convolutional
filters. Maybe Shrinkage layer?} 
layer that takes care of \emph{deactivating} neurons. We also modify
existing loss functions to incorporate a new term that takes care of balancing
network size and generalization capability appropriately.

\textbf{Filter Layers: } Filter layers have weights in the range $[0,+\infty]$ and are usually placed after
linear and convolutional layers.
The \textit{Filter Layer} takes an input of size $\left(B \times C \times D_1
  \times \dots \times D_n\right)$, where $B$ is the batch size, $C$ the number
of features (or channels, in the case of convolutional layers), and $D$ any
additional dimension. This structure makes it compatible with fully connected
layers with $n=0$ or convolutional layers with $n=2$. Their crucial property is
a parameter $\theta \in \mathbb{R}^C$. The output is defined as follows: \vspace{-1em}
\begin{equation} Filter(I;\theta) = I \circ \max(0, \theta) \end{equation}
Where $\circ$ is the pointwise multiplication, and $\theta$ is expanded in all
dimensions to match the input size (except the second one since they are equal
by definition). It is easy to see that if for any $k$, if $\theta_k \leq 0$,
the $k^{\text{th}}$ input feature/channel is multiplied by zero and have no
influence on the output. If this happens, we say the Filter layer deactivates
the neuron. These disabled neurons/channels can be removed from the network
without changing its output. Before explaining how that is achieved, we explain
next how the weights of the Filter Layer are initialized and adjusted during
training.

\textbf{Training Procedure: } Once Filter layers are placed in a network and initialized (sampled from the Uniform$[0, 1]$ distribution),
%To train networks we need start with a substantially oversized network, then we
%insert \textit{Filter Layers}  (usually after every linear or convolutional
%layer except the last one) and we sample their weight from the
%$\text{Uniform}(0, 1)$ distribution. 
we could train the network directly using our standard loss function, and we could achieve performance equivalent to a normal neural
network. However, our goal is to find the smallest network with reasonable
performance. We achieve that by introducing sparsity in the parameters of the
\textit{Filter Layers}, thus forcing the deactivation of neurons%. Indeed, having a negative component in the $\theta$
%parameter of the filter layer permamently disable its associated feature
%\gl{Maybe redundant ? we talked about that in the previous paragraph} . 
. To obtain this sparsity, we simply redefine the loss function:
\vspace{-.5em}
\begin{equation}
  L'(x,y;\theta) = L(x, y) + \lambda|\max(0, \theta)|
\end{equation}

The additional term $\lambda|\max(0, \theta)|$ introduces sparsity (see Lasso
loss~\cite{Tibshirani1996}). 
% The $\lambda$ parameter, that can take any
% positive value, adjusts how aggressively the network deactivates neurons, with
% larger values indicating more aggressive deactivation.
 The second component of the loss increases the gradient with respect to $\theta$, thus pushing its value towards zero. Neurons with little impact
on the original loss (gradient lower than $\lambda$), will not be able to
compete against this attraction towards zero. Because the entries in $\theta$
with a value of $0$ or less correspond to dead neurons, $\lambda$ effectively
controls the number of neurons/channels in the entire network. We introduced
the $\max(\dots)$ into the loss to make sure that neurons are permamently disabled
when performing gradient descent based optimization. Next, we
explain how to implement ShrinkNets efficiently.

\subsection{Strategies to Remove Neurons}

\textbf{Implementation}
It is possible to reduce the overhead of the training process by removing
neurons as soon as they become deactivated by $\theta$ going to 0.
%If
%disabled neurons are not quickly removed, the overhead might cause the training
%process to be significantly slower than classic neural networks. 
To do this, we implemented a \emph{neural garbage collection} mechanism which
prunes deactivated neurons on-the-fly,  reducing the processing time
and memory overhead. To support this feature, it is crucial to understand the
information flow between neurons and layers in the neural network. We achieve
this by representing such information flow as a graph. Vertices represent layers,
and edges are event-hubs responsible for propagating information about disabled
neurons to the relevant layers.
%!TEX root=paper.tex
%\subsection{Relation to Group Sparsity}
\noindent\textbf{Relation to Group Sparsity (LASSO): } \shrink removes neurons,
i.e., inputs and outputs of layers. For a fully connected layer defined as:
%
\begin{equation} \label{fully_connected}
  f_{\bm{A}, \bm{b}}(\bm{x})=a(\bm{Ax + b})
\end{equation}
%
where $\bm{A}$ represents the connections and $\bm{b}$ the bias,
removing an input neuron $j$ is equivalent to having $\left(\bm{A}^T\right)_j =
\bm{0}$. Removing an output neuron $i$ is the same as setting $\bm{A}_i = \bm{0}$
and $\bm{b}_i = 0$. Solving optimization problems while trying to set entire
group of parameters to zero is the goal of group sparsity regularization
\cite{Scardapane2017}. 
In  any partitioning of the set of parameters $\bm{\theta}$ defining a model in $p$
groups: $\bm{\theta} = \bigcup_{i=1}^P \bm{\theta}_i$, group sparsity 
penalty is defined as (where $\lambda$ is the regularization parameter): 
%
\vspace{-0.1in}
\begin{equation}
    \label{full_def}
  \Omega_\lambda^{gp} = \lambda \sum_{i=1}^p \sqrt{\mathbf{card}(\bm{\theta_i}}) \norm{\bm{\theta_i}}_2 \\
\end{equation}
\vspace{-0.2in}
% \srm{define notation -- what is $\lambda$;  what is $\#\theta$, what is $\Omega$? Where does the square root come from?} \gl{This is how it is defined in the
% original paper, it would require a few sentences to explain that, do really
% need to do ?}

In fully-connected layers, the groups are either: columns of
$\bm{A}$ if we want to remove inputs, or rows of $\bm{A}$ and the corresponding
entry in $\bm{b}$ if we want to remove outputs. For simplicity, we focus
our analysis in the simple one-layer case. In this case, filtering outputs does
not make sense, so we only consider removing inputs. The
group sparsity regularization then becomes (when $\sqrt{n}$ is folded into the $\lambda$)
%
\vspace{-0.1in}
\begin{equation} \label{group_sparsity_regularization}
  \Omega_\lambda^{gp} = \lambda \sum_{j=1}^p \norm{\bm{\left(A^T\right)_j}}_2 \\
\end{equation}
\vspace{-0.2in}

% Because $\forall i, \#\bm{\theta}_i = n$, embedded $\sqrt{n}$ inside $\lambda$.
%  \ra{is \# the general way of expressing
% cardinality? why not $|x|$?,}  \gl{In europe |x| is for the abolute value, and I was using it in the notation section that someone removed, therefore there would have been a notation conflict}to make the notation simpler, we

Group sparsity and \shrink try to achieve the same goal. We discuss next how
they are related to each other. First let's recall the two problems. In the
context of approximating $\bm{y}$ with a linear regression from features
$\bm{x}$, the two problems are:%
\vspace{-0.25in}
\begin{multicols}{2}
    \begin{equation*}
        \text{\shrink: } \min_{\bm{A}, \bm{\beta}} \norm{\bm{y} - \bm{A}\diag{\bm{\beta}}\bm{x}}_2^2 + \lambda \norm{\bm{\beta}}_1
    \end{equation*}
    \break
    \begin{equation*}
        \text{\textbf{Group sparsity: }}\min_{\bm{A}} \norm{\bm{y} - \bm{A}\bm{x}}_2^2 + \Omega_\lambda^{gp}
    \end{equation*}
\end{multicols}
\vspace{-0.1in}

We can prove that under the condition: $\forall j\in \intint{1, p},
\norm{\left(\bm{A}^T\right)_j}_2 = 1$ the two problems are equivalent by taking
$\bm{\beta}_j = \norm{\left(\bm{A}^T\right)_j}_2^2$, and replacing $\bm{A}$ by
$\bm{A}\left(\diag{\bm{\beta}}\right)^{-1}$. However, if we relax this constraint
then \shrink becomes non-convex and has no global minimum. The latter is true
because one can divide $\bm{\beta}$ by an arbitrarly large constant and
multipliying $A$ by the same value. Fortunately, by adding an extra term to the \shrink
regularization term we can avoid that problem and prove that:
%
\begin{equation}
  \min_{\bm{A}, \bm{\beta}} \norm{\bm{y} - \bm{A}\diag{\bm{\beta}}\bm{x}}_2^2 + \Omega_\lambda^s + \lambda_2\norm{A}_p^p
\end{equation}
%
has global minimums for all $p>0$. More specifically there are at least $2^k$,
where $k$ is the total number of components in $\bm{\beta}$. Indeed, for any
solution, one can obtain the same output by flipping any sign in $\bm{\beta}$
and the corresponding entries in $\bm{A}$.  This is the reason we defined the
regularized \shrink penalty above in \cref{full_def}.
In practice, we observed that $p=2$ or $p=1$ are good a choice; note that the latter
will also introduce additional sparsity into the parameters because the $l_1$ is, 
thest best convex approximation of the $l_0$ norm.
%!TEX root=paper.tex
\section{Evaluation}

The goal of our evaluation is to explore (1) whether, by varying $\lambda$,
\shrink can efficiently explore (in terms of number of training runs)  the
spectrum of high-accuracy models from small to large, on both CNNs and fully
connected networks.  Our results show that, for each network size, we obtain
models that perform as well or better than \textit{Static Networks}, trained via
traditional hyperparameter optimization;  (2) whether, because these  smaller
networks are dense, they result in improved inference times on both CPUs and
GPUs; and (3) whether the \shrink approach results in network architectures that
are substantially different than the best network architectures (in terms of
relative number of neurons per layer) identified in the literature.

\noindent\textbf{Implementation: }We implemented \swls and
the associated training procedure as a library in
pytorch~\cite{paszke2017automatic}. The layer can be freely mixed with other
popular layers such as convolutional layers, batchnorm layers, fully connected
layers, and used with all the traditional optimizers. We use our implementation
to evaluate \shrink throughout the evaluation section.

\subsection{Can \shrink achieve good accuracy?}
%\subsection{Performance vs. Traditional Methods}

To answer this question we compare \shrink with a traditional network. In both
cases, we need to perform hyperparameter optimization to explore different
network architectures. We perform random search, which is an effective technique
for this purpose \cite{BergstraJAMESBERGSTRA2012}. We evaluate \shrink on two
well-known datasets. One for which it is not possible to explore the entire
space of network architectures (\texttt{CIFAR10}) and one for which it is
possible to do so (\texttt{COVERTYPE}).

\noindent\textbf{Setup: } We assume no prior knowledge on the optimal batch
size, learning rate, $\lambda$ or weight decay ($\lambda_2$). Instead, we
trained a number of models, randomly and independently selecting the values of
these parameters from a range of reasonable values. Training is done using
gradient descent and the \textit{Adam} optimizer
\cite{DBLP:journals/corr/KingmaB14}. Specifically, we start with randomly
sampled learning
rate; for every $5$ epochs of non-improvement in validation
accuracy we divide the learning rate by $10$. We stop training after $400$
epochs or when the learning rate is under $10^{-7}$, whichever comes first. For
each of the models we trained, we pick the epoch with the best validation
accuracy and report the corresponding testing accuracy. Because of the nature of
our method, it can happen that for networks that are aggressively compressed,
the best validation accuracy is obtained early in training, before the size has
converged. To be sure that accuracy measured corresponds to the final shape and
not the starting shape, we only consider the second half of the training when
picking the best epoch. For each model, we also measure the total size, in terms
of number of floating point parameters, excluding the \swls because as described
in \cref{neuron_killing}, these are eliminated after training.


\subsubsection{Large Network Setting: \texttt{CIFAR10}}


\texttt{CIFAR10} is an image classification dataset containing $60000$ color
images $(3 \times 32 \times 32)$, belonging to $10$ different classes. We use it
with the \texttt{VGG16} network \cite{Srivastava2014}, which consists of
alternating convolutional layers and \textit{MaxPool} layers interleaved by
\textit{BatchNorm} \cite{DBLP:journals/corr/IoffeS15} and \textit{ReLU}
\cite{Nair2010} layers. The two last layers are fully connected layers
separated by a \textit{ReLU} activation function.

We applied \shrink to the VGG16 network by adding \swls
after each \textit{BatchNorm} layer and each fully connected layer (except the
last). Recall that \shrink assume that the starting size of the network is
an upper bound on the optimal size. Thus, we started with a
network with 2x the recommended size for each layer as an upper bound (this
is larger than what ImageNet uses). 

We compare against classical (\textit{Static}) networks. In such networks, the
number of parameters that control the size is large: 13 parameters for the
convolutional layers and $2$ for the fully connected layers. \shrink
effectively fuse all these parameters in a single $\lambda$, but in conventional
architectures where all of these parameters are free, it is infeasible to obtain
a reasonable sample of a search space of this size. For this reason, we rely on
the conventional heuristic that the original VGG architecture (and many CNNs)
use, where the number of channels is doubled every after \textit{MaxPool} laxer.
For \textit{Static Networks} we sample the size between $0.1$ and $2$ times the size
original one, optimized for ImageNet. We report the same numbers as we did for
\shrink and we compare the two distributions. 

The results are shown in the top figure of \cref{figure_CIFAR10}, with blue dots
indicating models produced by \shrink and orange dots indicating static networks. 
For each
model, we plot its accuracy and model size. The lines show the Pareto frontier
of models in each of the two optimization settings. \shrink explore the
trade-off between model size and accuracy more effectively. 

Note that the best performing \shrink model has $92.07\%$ accuracy which
is identical to the accuracy of the static network, while the \shrink model
is $2.22$ times smaller. In addition, if we give up just 1\% error, \shrink
find a model that is 35.5 times smaller than any static network that performs
as good or better.

\subsubsection{Small Network Setting: \texttt{COVERTYPE}}

The \texttt{COVERTYPE} \cite{Blackard:1998:CNN:928509} dataset contains $581012$
descriptions of geographical area (elevation, inclination, etc...) and the goal
is to predict the type of forest growing in each area. We picked this dataset
for two reasons. First it is simple, such that we can reach good accuracy with
only a few fully-connected layers. This is important because we want to show
that \shrink find sizes as good as \textit{Static Networks}, even if
we are sampling the entire space of possible network sizes. Second, Scardapane
et al~\cite{Scardapane2017} perform their evaluation on this dataset, which
allows us to compare the results obtained by our method with the method in
~\cite{Scardapane2017}.

%The experimental setup on this dataset is similar to {\tt CIFAR10}. 
We compare \shrink against the same architecture
used in \cite{Scardapane2017}, i.e., a three fully-connected layers network with no
\textit{Dropout} \cite{Srivastava2014} and no \textit{BatchNorm}. 
%\gl{Should we say
%here that we don't expect Dropout to work here ? I could write an entire
%paragraph about it if needed}. 
In this case, for the \textit{Static Networks}, we independently sample the
sizes of the three different layers to explore all possible architectures.

The results are shown in the top figure of \cref{figure_COVER}, with the two
optimization methods plotted as before. Here, {\it Static} method finds models
that perform well at a variety of sizes, because it is able to explore the
entire parameter space.  This is as expected;  the fact that \shrink perform
as well as the Static indicates that \shrink are doing an effective job of
exploring the parameter space using just the single $\lambda$ parameter.

Note that the best performing \shrink models has $96.91\%$ accuracy while the
best static model is only $96.66\%$ accurate, while the \shrink shrink model is $2.51$
times smaller. In addition, if we give up just 0.5\% error, \shrink find a
model that is 38.6 times smaller than any static network with equivalent accuracy.


\subsubsection{Summary}

We  demonstrated that it is possible to achieve networks with good accuracy
when using \shrink both when the network space cannot be explored entirely
(\texttt{CIFAR10}) and when it can, e.g., \texttt{COVERTYPE}. The most important
result is not that \shrink find networks of good accuracy, but that those
networks are much smaller than those found by a static method. The impact of the
network size on inference time is the subject of our next evaluation goal.


\subsection{Can ShrinkNets speed up inference?}

The previous experiment showed that \shrink find networks of similar or better accuracy
than static networks that are much smaller. We now explore if the reduction in size
translates into an improvement of the inference time.

As noted in the introduction, for some applications, compact models that offer
fast inference times are as important as absolute accuracy. 

%This observation motivates our 
% the experiment approach described in the previous section 
% and shown in the top of Figure~\ref{figure_CIFAR10} 
% and~\ref{figure_COVER}:
% for a  desired target accuracy, the Pareto optimal line shows the smallest
% network
% that satisfies achieves a given accuracy. 

In this section, we study the relationship between accuracy, network size and
inference time.  To do this, we select the smallest model that achieves a given
accuracy for the both \shrink and Static approach.  For each model, we measure
the time to run inference with the model.  We then compute the ratio of the
network size and inference time between \shrink and Static at each accuracy
level, and plot them on the bottom of Figure~\ref{figure_CIFAR10}
and~\ref{figure_COVER}.  We limit our plots to the models with $80-100\%$
accuracy range because those are the ones that we consider to be practically
useful.

The middle plot in each figure shows the ratio of model size between \shrink
and Static (values $>$1 mean \shrink are smaller) at different accuracy levels.
These figures show that is that size improvements are are particularly
significant for  \texttt{CIFAR10}. In the range of accuracies we are interested
in, improvements in size go from 4x to 40x. On the \texttt{COVERTYPE} dataset,
the compression ratio is always above 1 but it rarely exceeds 3x, except for
very high accuracies where \shrink find excellent, small solutions.  The fact
that the  \texttt{COVERTYPE} networks are not dramatically smaller is expected:
as the distribution at the top of Figure~\ref{figure_COVER} shows, the static
method is able to explore most of the parameter search space, so finds a range
of models that perform well at different sizes.

For speedup, we experimented with both CPUs and GPUs, and with different batch
sizes, where batch size indicates the number of inputs simultaneously fed to the
model for inference.  For each data set/GPU/CPU combination, we show results
with batch size 1, as well as with a batch size large enough to fully utilize
the hardware on each dataset and hardware configuration.  For example, for {\tt
CIFAR10} on CPU, a batch size of 64 fully utilizes the CPU, whereas a GPU can
execute many more models in parallel, so we use a larger batch size of 1024.
For {\tt COVERTYPE}, because the model is so much smaller, larger batches are
needed to fully utilize the hardware.  Note that when using a batch size of $1$
on GPU, we do not expect to (and do not) observe any improvement because
inference times are very small (typically about 10 $\mu s$), such that setup
time dominates overall runtime.  

The bottom four graphs in each figure show the results.  Again, the {\tt
CIFAR10} results show the benefit of the \shrink approach most dramatically.  On
CPU, speedups range up to 6x depending on the batch size, with many models
exceeding 3x speedup. In general, speedups are less than compression ratios, due
to overheads in problem setup, invocation, and result generation  in
Python/PyTorch.  On GPU, the speedups are less substantial because the CUDA
benchmarking utility that we use for evaluation can choose better algorithms for
larger matrices which masks some of our benefit, although they are still often
1.5x--2x faster for large batch sizes.  The speedup results on {\tt COVERTYPE}
are similar to those for network size:  because the networks are not much
smaller, they are not much faster either.

A key takeaway of these speedup results is that, unlike local sparsity
compression methods, our methods' improvement on size translates directly to
higher throughput at inference time~\cite{Han2015}.


\begin{figure*}[t]\centering
\begin{minipage}{2.7in}
\centering
\includegraphics[width=\columnwidth]{CIFAR10_VGG_summary-arrows}
\vspace*{-10mm}
\caption{\label{figure_CIFAR10} Summary of the result of random
search over the hyper-parameters the \texttt{CIFAR10} dataset}
\vspace*{-5mm}
\end{minipage}
\begin{minipage}{.3in}
~~
\end{minipage}
\begin{minipage}{2.7in}
\centering\includegraphics[width=\columnwidth]{COVER_FC_summary-arrows}
\vspace*{-10mm}
\caption{\label{figure_COVER} Summary of the result of random
search over the hyper-parameters the \texttt{COVERTYPE} dataset
}
\vspace*{-5mm}
\end{minipage}
\end{figure*}


\begin{figure}[htb]
\begin{center}
\vspace{-.1in}
\includegraphics[width=.6\columnwidth]{size_evolution}
\vspace*{-5mm} 
\caption{ Evolution of the size of
  each layer over time (lighter: beginning, darker: end). On top a very large
  network performing $92.07\%$, at the bottom a simpler model with $90.5\%$
  accuracy. 
} 
\label{fig:network_size_evolution}
\end{center}
\vspace*{-4mm}
\end{figure}

\subsection{Architectures obtained after convergence}
\shrink effectively explore the frontier of model size and accuracy. For a
given target accuracy, the size needed is significantly smaller than when we use the
"channel doubling" heuristic commonly used to size convolutional neural networks.
This suggests that this conventional heuristic may not in fact be optimal,
especially when looking for smaller models.  Empirically we observed this to
often be the case.  For example, during our experimentations on the
\texttt{MNIST} \cite{Lecun1998} and \texttt{FashionMNIST} \cite{Xiao2017}
datasets (not reported here due to space constraints), we observed that even
though these datasets have the same number of classes, input features, and
output distributions, for a fixed $\lambda$ \shrink converged to
considerably bigger networks in the case of \texttt{FashionMNIST}. This evidence
shows that optimal architecture not only depends on the output distribution or
shape of the data but actually reflects the dataset.  This makes sense, as
\texttt{MNIST} is a much easier problem than \texttt{FashionMNIST}.

To illustrate this point on a larger dataset, we show two examples of
architectures learned by \shrink in
Figure~\ref{fig:network_size_evolution}.  The right arrow shows the model with the
best test accuracy, with identical performance to the best static
network; the left arrow shows a network that slightly under-performs the best in
terms of accuracy but is significantly smaller that the best equivalent
\textit{Static Network}.  In the plot, the dashed line
shows the number of neurons in each layer of the original VGG net, and the
shaded regions show the size of the \shrink as it converges (with the darkest
region representing the fully converged network).  Observe that the final
network that is trained looks quite different in the two cases, with the optimal
performing network appearing similar to the original VGG net, whereas the
shrunken network allocates many fewer neurons to the middle layers, and then
additional neurons to the final fewer layers.



%!TEX root=paper.tex
\section{Related Work}

%\begin{itemize}
%  \item post-training compression techniques -- brain damage , 
%  \item group sparsity e.g., \cite{Scardapane2017} and non-parametric neural networks -- 
%  \item training dynamics paper: first overfitting and then randomization?, \gl{Here is the ref, if you can introduce it in the flow \cite{Shwartz-Ziv2017}}
%\end{itemize}

There are several lines of work related to optimizing network structure. 

%Given the importance of network structure, many researchers have explored the
%problem of finding the best network structure for a given learning task.  The
%proposed techniques broadly fall into five categories: random search andbrute
%force search, hyperparameter optimization, model compression after training,
%resizing models during training, and automated architecture search methods.

\noindent\textbf{Hyperparameter optimization techniques: }
One way to optimize network architecture is to use 
hyperparameter optimization.  Although many methods have been 
proposed, e.g., \cite{BergstraJAMESBERGSTRA2012,Snoek12},
randomized search has been shown to work surprising well.
%  Brute force search of network sizes is
%also become more practical due to faster and more powerful
%hardware~\cite{molchanov2016pruning}.  
The more complex methods for
hyperparameter optimization include techniques, e.g., ~\cite{Snoek12} typically
 select hyperparameter combinations that come from uncertain areas of the
hyperparameter space to search efficiency. 
As a generalization of this,  methods based on bandit algorithms (e.g.
~\cite{li2016hyperband, jamieson2016}) have also become a popular way to tune
hyperparameters by quickly discarding 
model configurations that perform badly. 
Although these methods could in theory be used to tune the number of neurons per layer
of a network, in practice no related work proposes this, because treating each layer as a hyperparameter
would lead an excessively large search space.
In contrast, with ShrinkNets the size of the network can be tuned with 
a single parameter, as we explain in the next section..
% More importantly, none of the hyperparameter optimization methods focuses on
% finding small networks, which is a crucial property of ShrinkNets, necessary to
% achieve good inference times.

%As noted before, all of the above techniques require many tens
%to hundreds of models to be trained, making this process computationally
%inefficient and slow.  More practically, the hyperparameter optimization
%literature does not evaluate their methods on network size and instead focuses
%on optimization hyperparameters such as learning rates and weight decay
%parameters.

\noindent\textbf{Model Compression: }Model compression techniques focus on
reducing the model size \emph{after} training, in contrast to ShrinkNets, which
reduces it \emph{while} training. 
Optimal brain damage~\cite{Cun} identifies connections in a network that are
unimportant and then prunes these connections.
DeepCompression~\cite{han2015deepcompression} takes this one step further and in
addition to pruning connections, it quantizes weights to make inference
extremely efficient.  A different vein of work such as ~\cite{romero2014fitnets,
hinton2015distilling} proposes techniques for distilling a network into a
simpler network or a different model. Because these techniques work after
training, they are orthogonal and complementary to ShrinkNets. Further,
some of these techniques, e.g.,~\cite{Han2015,Cun}, produce sparse matrices that
are not likely to improve inference times even though they reduce network size.
%Unlike our technique which works during
%training, these techiques are used after training and it would be interesting to
%apply them to ShrinkNets as well. 
%\cite{Abadi2016b} share the common goal of
%removing entire blocks of parameter to maintain dense matrices, however their
%method only applies to convolutional layers.

%\noindent\textbf{Auto-ML: } Some work focuses on automatically learning
%model architecture through the use of genetic algorithms and reinforcement
%learning techniques~\cite{DBLP:journals/corr/ZophL16, zoph2017learning}. These
%techniques are focused on learning higher-level architectures (e.g., building
%blocks for neural network architectures). In particular, they require to train
%full models and may take weeks to converge. 

%do not
%focus on finding small but well-performing networks for inference, which is the
%goal of ShrinkNets.
%\tim{Argument is not really convincing, but those techniques require to train
%full models and might take weeks to converge. }

\noindent\textbf{Dynamically Sizing Networks: }The techniques closest to our
proposed method are those based on group sparsity such as
~\cite{Scardapane2017}, and those like~\cite{Philipp} that dynamically grow and shrink
 the size of the network during training.  \cite{Scardapane2017}
presents a method that also deactivates neurons using a loss function based on
group-sparsity.  However, the exact details of how their method works are not
given, and their experimental results (on a small, fully connected network), are
substantially worse than ours as shown in Section~\ref{sec:experiments}.
\cite{Philipp} propose a method called Adaptive Radial-Angular Gradient Descent
that adds neurons on the fly and removes neurons via an $l_2$ penalty.  However,
this approach requires a new optimizer and takes longer to converge compared to
ShrinkNets.




\section{Discussion}

\gl{Did not have time to write it but added some pointers}

\begin{itemize}
  \item Where does the method shine:  We shine where there are too many layers
    To explore the network size and we have to rely on heuristics. It shines by
    it simplicity compared to other methods \gl{Do we want to talk about the
      number of lines of codes ?}. Prunning while we train, from the compression
    results seem to be benificial to prune while training to achieve very good
    compression.
  \item Side-effects of method: smaller networks, time to train. There is not
    significant difference in training time, mostly because the filter layer is
    not optimized. However we can say the convergence speed is improved. For
    example
  \item Potential extensions/limitations: We could add a filter that learn the
    size of convolutions for convolution layers. Supporting Residual Network
    would allow us to support more architecture. We could also try to mix try to
    learn the number of layers such is described in~\cite{meier}
\end{itemize}
%!TEX root=paper.tex
\section{Conclusion}

We presented \shrink, an approach to learn deep network sizes while training.
\shrink employs a \swl, which deactivates neurons, as well as
of a method to remove them, which reduces network sizes, leading to faster
inference times. We demonstrated these claims on on two well-known datasets, on
which we achieved networks of the same accuracy as traditional neural
networks, but up to 35X smaller, with inference speedups of up
to 6X.



\appendix
%!TEX root=paper.tex
\section{Appendix}

\begin{figure}[t]
\begin{center}
\includegraphics[width=.7\columnwidth]{regressions}
\vspace*{-5mm}
\caption{\label{sparsity_accuracy}Loss/Sparsity trade off comparison between Group Sparsity and Shrinknet on linear and logistic regression. From top to bottom and left to right we show the results for \texttt{scm1d}, \texttt{oes97}, \texttt{gina\_prior2} and \texttt{gsadd}.}

\end{center}
\vspace*{-4mm}
\end{figure}

\subsection{Proofs}
Unless specified, all the proofs consider the Multi-Target linear regression problem
\begin{proposition}
\label{gps_equivalence}
  $\forall (n, p) \in \mathbb{N}_+^2, \bm{y} \in \mathbb{R}^{n}, \bm{x} \in \mathbb{R}^{p} \lambda \in \mathbb{R}$
  
  \begin{align*}
    & \min_{\bf{A}} \norm{\bf{y} - \bf{A}\bf{x}}_2^2 + \lambda \sum_{j=1}^p \norm{\left(A^T\right)_j}_2 \\
     = &\min_{\bf{A'}, \bm{\beta}} \norm{\bf{y} - \bf{A'}\diag{\bm{\beta}}\bf{x}}_2^2 + \lambda \norm{\bm{\beta}}_1 \\
     & \text{s.t.} \forall j \in \intint{1, p}, \norm{\left(A'^T\right)_j}_2^2 = 1
  \end{align*}
\end{proposition}

\begin{proof}
  In order to prove this statement we will show that for any solution $\bm{A}$ in the first problem, there exists a solution in the second with the exact same value, and vice-versa.
We now assume we have a potential solution $\bm{A}$ for the first problem and we define $\bm{\beta}$ such that $\bm{\beta}_j = \norm{\left(\bm{A}^T\right)_j}_2^2$, and $\bm{A}' = \bm{A}\left(\diag{\bm{\beta}}\right)^{-1}$. It is easy to see that the constraint on $\bm{A}'$ is statisfied by construction. Now:
  \begin{align*}
    & \norm{\bf{y} - \bf{A}\bf{x}}_2^2 + \lambda \sum_{j=1}^p \norm{\left(A^T\right)_j}_2 \\
    = &\norm{\bf{y} - \bf{A'}\diag{\bm{\beta}}\bf{x}}_2^2 + \lambda \sum_{j=1}^p \norm{\left(A'^T\right)_j\beta_j}_2 \\
    = &\norm{\bf{y} - \bf{A'}\diag{\bm{\beta}}\bf{x}}_2^2 + \lambda \sum_{j=1}^p \abs{\beta_j} \cdot 1\\
    = &\norm{\bf{y} - \bf{A'}\diag{\bm{\beta}}\bf{x}}_2^2 + \lambda \norm{\bm{\beta}}_1
  \end{align*}
  Assuming we take an $\bm{A}'$ that satisfy the constraint and a $\bm{\beta}$, we can define $\bm{A} = \bm{A'}\diag{\bm{\beta}}$. We can apply the same operations in reverse order and obtain an instance of the first problem with the same value. We can now see that the two problems must have the same minimum otherwise we would be able to construct a solution to the other with exact same value.
\end{proof}

\begin{proposition}
\label{unconstrained_non_convex}
\begin{equation*}
     \norm{\bf{y} - \bf{A}\diag{\bm{\beta}}\bf{x}}_2^2
\end{equation*}
is not convex in $\bm{A}$ and $\bm{\beta}$.
\begin{proof}
  To prove this we will take the simplest instance of the problem: with only scalars. We have $f(a, \beta) = \left(y - a\beta x\right)^2$. For simplicty let's take $y = $ and $x > 0$. If we take two candidates $s_1 = (0, 2)$ and $s_2 = (2, 0)$, we have $f(s_1) = f(s_2) = 0$. However $f(\frac{2}{2}, \frac{2}{2}) = x > \frac{1}{2} f(0, 2) + \frac{1}{2}f(2, 0)$, which break the convexity property. Since we showed that a particular case of the problem is non-convex then necessarly the general cannot be convex.
\end{proof}
\end{proposition}

\begin{proposition}
\label{unconstrained_shrinknet_no_min}
\begin{equation*}
     \min_{\bf{A}, \bm{\beta}} \norm{\bf{y} - \bf{A}\diag{\bm{\beta}}\bf{x}}_2^2 + \lambda \norm{\bm{\beta}}_1
\end{equation*}
has no solution if $\lambda > 0$.
\end{proposition}
\begin{proof}
  Let's assume this problem has a minimum $\bm{A}^*, \bm{\beta}^*$. Let's consider $2\bm{A}^*, \frac{1}{2}\bm{\beta}^*$. Trivially the first component of the sum is identical for the two solutions, however $\lambda\norm{\frac{1}{2}\bm{\beta}} < \lambda\norm{\bm{\beta}}$. Therefore $\bm{A}^*, \bm{\beta}^*$ cannot be the minimum. We conclude that this problem has no solution.
\end{proof}
\begin{proposition}
  \label{shrinknet_regularized_minimum}
For this proposition we will not restrict ourselves to single layer but the composition of an an arbitrary large ($n$) layers as defined individually as $f_{\bm{A}_i, \bm{\beta}_i, \bm{b}_i}(x) = a(\bm{A_i}\diag{\bm{\beta_i}}\bm{x} + \bm{b_i})$. The entire network follows as: $N(\bm{x}) = \left(\bigcirc_{i=1}^n f_{\bm{A_i}, \bm{\beta_i}, \bm{b_i}}\right)(\bm{x})$. For $\lambda > 0$, $\lambda_2 > 0$ and $p > 0$ we have:
  \begin{equation*}
    \min \norm{\bm{y} - N(\bm{x})}_2^2 + \Omega_{\lambda, \lambda_2, p}^{rs}
  \end{equation*}
  has at least $2^k$ global minimum where $k = \sum_{i=1}^n \#\bm{\beta_i}$
\end{proposition}

\begin{proof}
First let's prove that there is at least one minimum to this problem. The two components of the expression are always positive so we know that this problem is bounded by below by $0$. Let's assume this function does not have a minimum. Then there is a sequence of parameters $(S_n)_{n>0}$ such that the function evaluated at that point convereges to the infimum of the problem. Since the function is defined everywhere does not have a minimum then this sequence must diverge. Since the entire sequence deverge the there is at least one individual parameter that diverges. First case, the parameter is a component $k$ of some $\bm{\beta_i}$ for some $i$. Necessarly $\norm{\bm{\beta_i}}_1$ diverge towards $+ \infty$, which is incompatible with the fact that $(S_n)$ converges to the infimum. We can have the exact same argument if the diverging parameter is in $\bm{A_i}$ or $\bm{b_i}$ because $p > 0$. Since there is always a contradiction then our assumption that the function has no global minimum must be false. Therefore, this problem has at least one global minimum.

\par Let's consider one optimal solution of the problem. For each component $k$ of $\bm{\beta_i}$ for some $i$. Negating it and negating the $k^{th}$ column of $\bm{A_i}$ does not change the the first part of the objetive because the two factors cancel each other. The two norms do not change either because by definition the norm is independant of the sign. As a result these two sets of parameter have the same value and are both global minimum. It is easy to see that going from this global minimum we can decide to negate or not each element in each $\bm{\beta_i}$. We have a binary choice for each parameter, there are $k = \sum_{i=1}^n \#\bm{\beta_i}$ parameters, so we have at least $2^k$ global minima.

\end{proof}
\subsection{Multi-Target Linear and Multi-Class Logistic regressions experiments}
As we showed, Group sparsity share similarities with our method, and we claim
that ShrinkNets are a relaxation of group sparsity.  In this experiment we want
to compare the two aproaches.  We decided to focus on multi-target linear
regression because in the single target case, groups in the Group Sparsity
problem would have a size of one ($\bm{A}$ would be a vector in this case).

The evaluation will be done on two datasets \texttt{scm1d} and \texttt{oes97}
\cite{Spyromitros-Xioufis2016} for linear regressions and we will use \texttt{gina\_prior2} \cite{4371065} and
the \textit{Gas Sensor Array Drift Dataset} \cite{VERGARA2012320} (that we shorten in
\texttt{gsadd}) for logistic regressions.

For each dataset we fit with different regularization parameters and measure
the error and sparsity obtained after convergence. In this context we define
sparsity as the ratio of columns that have all their weight under $10^{-3}$ in
absolute value. Regularization parameters were choosed in order to obtain the
widest sparsity spectrum. Loss is normalized depending on the problem to be in
the $[0, 1]$ range. We summarized the results in \cref{sparsity_accuracy}. From
our experiments it is clear that ShrinkNets can fit the data closer than Group
Sparsity for the same amount of sparsity. The fact that we are able to reach
very low loss demonstrate that even if our objective function is non convex, in
practice it works as good or better as convex alternatives.



% In the unusual situation where you want a paper to appear in the
% references without citing it in the main text, use \nocite
\nocite{OpenML2013}

\bibliography{custom}
\bibliographystyle{icml2018}



\end{document}
