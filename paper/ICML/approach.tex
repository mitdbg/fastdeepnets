\section{Our Approach}

% from the universality theorem, we know that we can find a network that ca
% arbitrarly fit a function. Therefore, for a given error tolreance there exists
% an optimal size to solve a given problem.

Our approach is based on the insight that given enough capacity (and time), an 
optimization routine can find a network that can fit an arbitrary function.
Therefore, by starting with an oversize network and applying on/off switches to
each of the neurons in it, we can identify a {\it pruned} version of the network
that can achieve similar accuracy to the oversized network.


% If we have an oversized network then there exist a pruned version that can still
% achieve our goal. The main idea is to consider an on/off switch for each neuron,
% and we want to find an assignement for this switches that achieve a certain
% size/accuracy tradeof. 
We model these on/off switches by multiplying each input
(or output) of each layer by a number $0$ or $1$. 
$0$ will deactivate the
neuron, $1$ will let the signal go through. 
We want to minimize the number of on
switches to reduce the model size as much as we can. 
This can be modeled solved
by jointly minimizing the objective of the network and a factor of the L0 norm
of the vector containing all the on/off switches.

Finding an optimal binary assignement is an NP-Hard problem. 
% (\textbf{Should we
% prove this ? I think it should be fairly doable reducing it to 3-SAT
% considering the structure of neural networks. No.})
We relax this problem
% problem because this is what people usually do with NP-Hard problem. Our
% relaxation is we allow 
by allowing $\bm{\theta}$ to be a real number instead of a 0/1 value, thus
constraining the L1 norm as opposed to the L0 norm.
% , effectively
% t We
% also use L1 instead of L1. This way we obtain a non-convex, but at least
% differentiable (almost everywhere).

This approach assumes that we start with an upper bound on the model size. This
obviously translates in a computational overhead. Our insight is that some
usless neurons (we have multiple definitions below) can be removed early without
impacting the final solution. It has two practical implications: It mitigate the
issue we describe but it also allows other neurons to adapt as soon as one of
their peer is killed. Existing technique usually remove them after convergence
and require an extra fine-tuning process to compensate for the removal.

% The key components of the system are: The filter vectors that simulate the
% continuous on/off switches, a regularization that tries to kill neurons, the
% neuron removal strategy that detects neurons that should probably be removed,
% and the garbage collection that effectively remove dead neurons from the model,
% and the simplifaction procedure that remove the filter vector for fast
% inference.

% We will describe these components in the upcoming sections.

% \subsection{Notations}

% \par In order to avoid any potential ambiguity, in this section we will
% describe in details the mathematical notations used in this article. 
% Non-bold
% letters represent scalar values, while bold lowercase and upper case
% repectively denote vectors and matrices. 
% $\bm{A}^T$ stands for the transpose of
% the matrix $\bm{A}$. 
% Subscripts are used to index particular elements of
% vectors and matrices. 
% $\bm{x}_i$, $\bm{A}_i$, $\left(\bm{A}^T\right)_j$ and
% $\bm{A}_{i,j}$ respectively correspond to the $i^{th}$ component of $\bm{x}$,
% the $i^{th}$ row of $\bm{A}$, the $j^{th}$ column of $\bm{A}$ and the $j^{th}$
% component of the $i^{th}$ row of $\bm{A}$. 
% All the following definitions assume
% $\bm{A}$ to be an $n\times p$ matrix.  
 
% For any $l \in \left[0, +\infty\right]$ we define the norm: $\norm{\bm{A}}_l =
% \left(\sum_{i=1}^n \sum_{j=1}^p \abs{\bm{A}_{i, j}}^l\right)^{\frac{1}{p}}$. 
% For the rest of this paper and unless stated otherwise, $\bm{y}$ will represent the output of a network, $\bm{x}$ the input, $\bm{b}$ a bias, $\lambda$ regularization factors, $\Omega$ regularization methods, $\bm{\theta}$ general model parameters and $a$ will stand for any element-wise activation function. 
% The only constraint that we want to enforce is that $a(0) = 0$. 
% We use $\intint{u, v}$ to denote inteveral of integers, 
% $\bm{0}$ is the null vector (size depending on the context). 
% $\#S$ is meant to represent the cardinality of a set $S$. 
% To simplify the notation of function composition we use the following operator: 
% $g(f(\bm{x})) = (f \circ g)(\bm{x})$ and for a long sequence of functions from 
% $f_1$ to $f_n$ we use: $f_n(...f_1(\bm{x})) = \left(\bigcirc_{k = 1}^n f_k\right)(\bm{x})$.

We implement our strategy of dynamically deactivating neurons by means of a 
specialized neural network layer called the \textbf{Switch Layer}.
We first describe the key concepts and theory backing the switch layer and then
provide a brief summary of our implementation of switch layers.


\subsection{The Switch Layer}

% \textbf{Do you like the name of the layer ? Yes!}

Let $L$ be a layer in a neural network that takes an input tensor $\bm{x}$ and 
produces an output tensor $y$ of shape $\left(c \times d_1 \times \dots \times 
d_n\right)$ where $c$ is the number of neurons in that layer.
For instance, for fully connected layers, $n$=0 and the output is single 
dimensional of size $c$ (ignoring batch size for now) while for a 2-D 
convolutional layer, $n$=2 and $c$ is the number of output channels or feature 
maps.
% maps
% For instance, for fully connected layers, $n$=0 and the output is single 
% dimensional of size $c$ (ignoring batch size for now) while for a 2-D 
% convolutional layer, $n$=2 and $c$ is the number of output channels or feature 
% maps.
%  and furthermore, let the output
% of layer $l$ be a tensor of shape $\left(C \times D_1 \times \dots \times 
% D_n\right)$ where $C$ is the number of neurons in $l$.
% In the case of a fully connected layer, $C$ is the dimensionality of the output
% of the layer whereas in the case of a convolutional layer, $C$ is the number of
% feature maps and $\left(D_1 \times \dots \times D_n\right)$ is the size of those feature
% maps.

Suppose we wish to tune the size of $L$ by applying a switch layer.
A switch layer $S$ applied to the output of $L$ can be parametrized by a 
vector $\theta \in \mathbb{R}^c$ such that the result of applying $S$ to $L(\bm{x})$
is a tensor also of size $\left(c \times d_1 \times \dots \times d_n\right)$
such that: 
\begin{equation} 
S_{\theta}(L(\bm{x}))_{i,...} = \bm{\theta}_iL(\bm{x})_{i, ...}
\end{equation}
% where $\diag{\bm{\theta}}$ is a $c\times c$ matrix such that: $\forall
% 1 \leq i \leq c$, $\diag{\bm{\theta}}_{i, i} = \bm{\theta}_i$ and $0$ otherwise. 
Effectively, once passed through the switch layer, each output channel $i$ 
produced by $L$ is scaled by the corresponding $\theta_i$.
If for any $k$, if $\theta_i = 0$, the $i^{\text{th}}$ channel is multiplied by 
zero and wont contribute to any computations after the switch layer.
If this happens, we say the Switch layer has {\it deactivated} or killed the 
neuron of layer $L$ corresponding to channel $i$. 
% Switch layers have weights in the range $[-\infty,+\infty]$ and are usually 
% placed after linear and convolutional layers.
% The \textit{Switch Layer} takes an input of size $\left(B \times C \times D_1
%   \times \dots \times D_n\right)$, where $B$ is the batch size, $C$ the number
% of features (or channels, in the case of convolutional layers), and $D$ any
% additional dimension. 
% This structure makes it compatible with fully connected
% layers with $n=0$ or convolutional layers with $n=2$. 
% Their crucial property is
% a parameter $\theta \in \mathbb{R}^C$. 
% The output is defined as follows: \vspace{-1em}
% \begin{equation} 
% Switch(\bm{I};\bm{\theta}) = \diag{\bm{\theta}} \bm{I}  
% \end{equation}
% where $\diag{\bm{b}}$ a $n\times n$ matrix such that: $\forall
% 1 \leq i \leq n$, $\diag{\bm{b}}_{i, i} = \bm{b}_i$ and $0$ otherwise. 

% Where $\theta$ is expanded in all
% dimensions to match the input size (except the second one since they are equal
% by definition). 

% These disabled neurons/channels can be removed from the network
% without changing its output. 
% Before explaining how that is achieved, we explain
% next how the weights of the Switch Layer are initialized and adjusted during
% training.

\subsection{Training Procedure} 

To tune the size of a network, we place Switch Layers after each layer whose size
we wish to tune; these are typically the fully connected and convolutional layers
in a network.
Since the switch layers are adept at deactivating neurons, we start with an
oversized network (i.e. network with more capacity than required) and then use
switch layers to deactivate or kill off neurons that are unnecessary.

Formally, we can express this procedure in terms of a sparsity constraint that pushes
values in the $\theta$ vector to 0.
Given a neural network parameterized by weights $\bm{W}$ and switch layer 
parameters $\bm{\theta}$, we optimize the following ShrinkNet loss that 
augments the regular training loss with a 
regularization term for the switch parameters and another on the network weights.
\begin{equation}
  L_{SN}(\bm{x},\bm{y};\bm{W}, \bm{\theta}) = L(\bm{x}, \bm{y}; \bm{W}) +
  \lambda\norm{\theta}_1 + \lambda_2\norm{\bm{W}}_p
\end{equation}

%To train networks we need start with a substantially oversized network, then we
%insert \textit{Switch Layers}  (usually after every linear or convolutional
%layer except the last one) and we sample their weight from the
%$\text{Uniform}(0, 1)$ distribution. 
% we could train the network directly using our standard loss function, and we could achieve performance equivalent to a normal neural
% network. However, our goal is to find the smallest network with reasonable
% performance. We achieve that by introducing sparsity in the parameters of the
% \textit{Switch Layers}, thus forcing the deactivation of neurons
%. Indeed, having a negative component in the $\theta$
%parameter of the filter layer permamently disable its associated feature
%\gl{Maybe redundant ? we talked about that in the previous paragraph} . 
% To obtain this sparsity, we simply redefine the loss function:
% \begin{equation}
%   L'(\bm{x},\bm{y};\bm{W}, \bm{\theta}) = L(\bm{x}, \bm{y}; \bm{W}) +
%   \lambda\norm{\theta}_1 + \lambda_2\norm{\bm{W}}_p
% \end{equation}

% The additional term $\lambda|\max(0, \theta)|$ introduces sparsity (see Lasso
% loss~\cite{Tibshirani1996}). 
% The $\lambda$ parameter, that can take any
% positive value, adjusts how aggressively the network deactivates neurons, with
% larger values indicating more aggressive deactivation.
%  The second component of the loss increases the gradient with respect to $\theta$, thus pushing its value towards zero. Neurons with little impact
% on the original loss (gradient lower than $\lambda$), will not be able to
% compete against this attraction towards zero. Because the entries in $\theta$
% with a value of $0$ or less correspond to dead neurons, $\lambda$ effectively
% controls the number of neurons/channels in the entire network. Without the last
% term our problem sounds very similar to the Group Sparsity regularization which
% is well known in the area of linear and logistic regressions. In the next
% section we will try to undercover the relationship between these two problems,
% explain why we need this additional regularization term and what should be the
% value of $p$.

%!TEX root=paper.tex
%\subsection{Relation to Group Sparsity}
\noindent\textbf{Relation to Group Sparsity (LASSO): } \shrink removes neurons,
i.e., inputs and outputs of layers. For a fully connected layer defined as:
%
\begin{equation} \label{fully_connected}
  f_{\bm{A}, \bm{b}}(\bm{x})=a(\bm{Ax + b})
\end{equation}
%
where $\bm{A}$ represents the connections and $\bm{b}$ the bias,
removing an input neuron $j$ is equivalent to having $\left(\bm{A}^T\right)_j =
\bm{0}$. Removing an output neuron $i$ is the same as setting $\bm{A}_i = \bm{0}$
and $\bm{b}_i = 0$. Solving optimization problems while trying to set entire
group of parameters to zero is the goal of group sparsity regularization
\cite{Scardapane2017}. 
In  any partitioning of the set of parameters $\bm{\theta}$ defining a model in $p$
groups: $\bm{\theta} = \bigcup_{i=1}^P \bm{\theta}_i$, group sparsity 
penalty is defined as (where $\lambda$ is the regularization parameter): 
%
\vspace{-0.1in}
\begin{equation}
    \label{full_def}
  \Omega_\lambda^{gp} = \lambda \sum_{i=1}^p \sqrt{\mathbf{card}(\bm{\theta_i}}) \norm{\bm{\theta_i}}_2 \\
\end{equation}
\vspace{-0.2in}
% \srm{define notation -- what is $\lambda$;  what is $\#\theta$, what is $\Omega$? Where does the square root come from?} \gl{This is how it is defined in the
% original paper, it would require a few sentences to explain that, do really
% need to do ?}

In fully-connected layers, the groups are either: columns of
$\bm{A}$ if we want to remove inputs, or rows of $\bm{A}$ and the corresponding
entry in $\bm{b}$ if we want to remove outputs. For simplicity, we focus
our analysis in the simple one-layer case. In this case, filtering outputs does
not make sense, so we only consider removing inputs. The
group sparsity regularization then becomes (when $\sqrt{n}$ is folded into the $\lambda$)
%
\vspace{-0.1in}
\begin{equation} \label{group_sparsity_regularization}
  \Omega_\lambda^{gp} = \lambda \sum_{j=1}^p \norm{\bm{\left(A^T\right)_j}}_2 \\
\end{equation}
\vspace{-0.2in}

% Because $\forall i, \#\bm{\theta}_i = n$, embedded $\sqrt{n}$ inside $\lambda$.
%  \ra{is \# the general way of expressing
% cardinality? why not $|x|$?,}  \gl{In europe |x| is for the abolute value, and I was using it in the notation section that someone removed, therefore there would have been a notation conflict}to make the notation simpler, we

Group sparsity and \shrink try to achieve the same goal. We discuss next how
they are related to each other. First let's recall the two problems. In the
context of approximating $\bm{y}$ with a linear regression from features
$\bm{x}$, the two problems are:%
\vspace{-0.25in}
\begin{multicols}{2}
    \begin{equation*}
        \text{\shrink: } \min_{\bm{A}, \bm{\beta}} \norm{\bm{y} - \bm{A}\diag{\bm{\beta}}\bm{x}}_2^2 + \lambda \norm{\bm{\beta}}_1
    \end{equation*}
    \break
    \begin{equation*}
        \text{\textbf{Group sparsity: }}\min_{\bm{A}} \norm{\bm{y} - \bm{A}\bm{x}}_2^2 + \Omega_\lambda^{gp}
    \end{equation*}
\end{multicols}
\vspace{-0.1in}

We can prove that under the condition: $\forall j\in \intint{1, p},
\norm{\left(\bm{A}^T\right)_j}_2 = 1$ the two problems are equivalent by taking
$\bm{\beta}_j = \norm{\left(\bm{A}^T\right)_j}_2^2$, and replacing $\bm{A}$ by
$\bm{A}\left(\diag{\bm{\beta}}\right)^{-1}$. However, if we relax this constraint
then \shrink becomes non-convex and has no global minimum. The latter is true
because one can divide $\bm{\beta}$ by an arbitrarly large constant and
multipliying $A$ by the same value. Fortunately, by adding an extra term to the \shrink
regularization term we can avoid that problem and prove that:
%
\begin{equation}
  \min_{\bm{A}, \bm{\beta}} \norm{\bm{y} - \bm{A}\diag{\bm{\beta}}\bm{x}}_2^2 + \Omega_\lambda^s + \lambda_2\norm{A}_p^p
\end{equation}
%
has global minimums for all $p>0$. More specifically there are at least $2^k$,
where $k$ is the total number of components in $\bm{\beta}$. Indeed, for any
solution, one can obtain the same output by flipping any sign in $\bm{\beta}$
and the corresponding entries in $\bm{A}$.  This is the reason we defined the
regularized \shrink penalty above in \cref{full_def}.
In practice, we observed that $p=2$ or $p=1$ are good a choice; note that the latter
will also introduce additional sparsity into the parameters because the $l_1$ is, 
thest best convex approximation of the $l_0$ norm.

\subsubsection{Neuron killing}
Once Switch layers are placed in a network and
initialized (sampled from the $\mathcal{N}(0, 1)$ distribution),

\textbf{Threshold strategy}:  We kill neurons based on a threshold (in absolute
value), this is the method used by deep compression. It is bad because it is
scale dependant and is not robust to noise in the gradients (explain that)
\\ \textbf{Sign change strategy}:  We look at when the sign change. This is also
sensitive to noise but has the advantage
\textbf{Sign variance strategy}: We measure the variance of the sign ($-1, 1$) of each component in the \textit{Switch Layer}, and consider it dead when it goes over a predefined threshold. \gl{Maybe we should add some intuition on why it makes sense}. \gl{We should probably explain the impact of $\gamma$ and report numbers from the expertiment that show that it does not hurt too much to kill neurons, we have the data in this experiment in the evaluation that we probably want to remove later.}

\subsection{Implementation}

We have implemented \textbf{Switch Layers} and the associated training procedure
as a library in pytorch~\cite{github}.
The layer can be freely mixed with other popular layers such as convolutional
layers, batchnorm layers, fully connected layers etc.
There are two key challenges that must be overcome while implementing 
\textbf{Switch Layers}.
First, since ShrinkNets kill off a large fraction of neurons, we must remove the
overhead of carrying around dead neurons.
And second, we must update the various network layers and training machinery (e.g.
optimizer state) to reflect the deactivation of neurons.
We address these challenges by augmenting every layer with additional information
about neuron status and implementing a \emph{neural garbage collection} 
mechanism which prunes deactivated neurons on-the-fly and updates the state of 
the training machinery.
% We do so by augmenting every layer with additional information such as what
% neurons have been deactivated.
% The layer then updates its size accordingly and this change is then propagated
% to other parts of the training machinery including the optimizer which must
% know the sizes of parameters.

A second implementation challenge we address with ShrinkNets is the potentially
higher inference time.
This increase in inference time is both due to the extra computation introduced 
by switch layers and because our switch layers are not currently optimized for 
CUDA.
Therefore, for now, our library provides a post-training routine to ``fold''
switch layers into existing, optimized layers.
Specifically, for every switch layer, we check if the parent or child of this 
layer is a linearly scalable layer (e.g. convolutional layer, fully connected layer,
batchnormalization layer) and if so, we fold the switch layer into this 
neighboring layer by multiplying through by the switch weights.
In the unlikely case that a switch layer is sandwiched between two non-linearly
scalable layers, we leave the switch layer as is.


% \subsection{Speeding up training}

% In order to have a practical implementation of ShrinkNets it is absolutely
% necessary to prune useless neurons as quick as possible. In order to try more
% different strategies we split this task in two subproblems: detecting
% potentially non-crucial neurons and setting their corresponding entry in the
% switch layer to exactly 0, and neural garbage collection that 



% In our library we implemented a variant of the second strategy and the last one.

% \subsubsection{Neural garbage collection}




% It is possible to reduce the overhead of the training process by removing
% neurons as soon as they become deactivated by $\theta$ going to 0.
%If
%disabled neurons are not quickly removed, the overhead might cause the training
%process to be significantly slower than classic neural networks. 
% To do this, we implemented a \emph{neural garbage collection} mechanism which
% prunes deactivated neurons on-the-fly,  reducing the processing time
% and memory overhead. 
% To support this feature, it is crucial to understand the
% information flow between neurons and layers in the neural network. 
% We achieve
% this by representing such information flow as a graph. 
% Vertices represent layers,
% and edges are event-hubs responsible for propagating information about disabled
% neurons to the relevant layers. 
% As soon as a layer receive an event, it updates
% his parameters to reflect the new size. 
% Each operation is stored in a log that
% is used later to update other component of the system. 
% For example some
% optimizers like Adam [Ref] store state about each parameter, their state needs
% to be updated the exact same way and at the exact same time as layers in order
% to obtain apropriate parameter updates.

% \subsubsection{Speeding up Inference}
% \mpv{This may be moved to discussion or ``other considerations'' or impl details.}



% our post-processing code searches the 
% network graph to identify a parent or child layer the linear layer closest (child or parent) to every 
% switch layer and multiplies the weights of that layer by the corresponding
% switch layer weights.


% Our proposed \textit{Switch Layer} are relevant only during training, and can be
% removed from the network architecture during inference if they can be ``folded''
% into the nearest linear layer.
% Specifically, we provide a post-processing routine that will, for each switch 
% layer, search the network graph 

% The sole puropose of \textit{Switch Layer} are to detect which neurons should be
% killed and which should be left. When the training is over we will never remove
% more neurons. Therefore, they loose all their interest at inference time. 
% To
% improve size and inference time we propose the following method to get rid of
% them without changing the output of the model (modulo floating point errors).
% The basic idea is to start from every \textit{Switch Layer} and try to find the
% nearset linear operator (child or parent) and multiply its weights by the values
% in the switch layer. For example we can merge Filter layers with surrounding
% Convolutional layer, Linear layer or even BatchNormalization [ref] if they are
% before the \textit{Switch Layer}. However we need to be extremely careful not to
% cross a non-linearity otherwise it would change the output of network.